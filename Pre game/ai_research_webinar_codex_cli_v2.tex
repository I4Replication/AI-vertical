
\documentclass[aspectratio=169,professionalfonts]{beamer}
\usetheme{default}

% Basic packages
\usepackage[utf8]{inputenc}
\usepackage{lmodern}
\usepackage{hyperref}
\usepackage{graphicx}
\usepackage{booktabs}
\usepackage{amsmath, amssymb}
\usepackage{array}
\usepackage{listings}
\usepackage{csquotes}
\usepackage{xspace}

% Beamer metadata
\title{AI Replication Games October 16th}
\subtitle{A one hour training for quantitative social sciences}
\author{Juan Pablo Posada Aparicio}
\institute{Institute for Replication, University of Ottawa}
\date{October 2025}

% Utilities
\newcommand{\feature}[1]{\textbf{#1}}
\newcommand{\demo}{\textbf{Demo. }}
\newcommand{\notepara}[1]{\small\textit{#1}}

\begin{document}

%-----------------------
\begin{frame}
  \titlepage
\end{frame}

%-----------------------
\begin{frame}{Session goals}
  \begin{itemize}
    \item Align on the Replication Games storyline, schedule, and roles before event day.
    \item Clarify randomization, treatment arms, and deliverables drawn from the preregistered design.
  \end{itemize}
\end{frame}

%-----------------------
\begin{frame}{Roadmap, about 60 minutes}
  \begin{itemize}
    \item 5 minutes. Pre-games context and storyline.
    \item 10 minutes. Study design, randomization, and assignments.
    \item 15 minutes. Event-day operations, deliverables, and support.
    \item 20 minutes. ChatGPT Plus feature tour across the research workflow.
    \item 5 minutes. Q and A.

  \end{itemize}
\end{frame}

%-----------------------
\section{Pre-Games Orientation}

\begin{frame}{Pre-games storyline}
  \begin{itemize}
    \item Kick off the Replication Games and align on the narrative before event day.
    \item Friendly human versus AI-augmented challenge tests speed, accuracy, and issue-spotting.
    \item We study vertical gaps across expertise tiers and horizontal gaps across disciplines.
  \end{itemize}
\end{frame}

\begin{frame}{Team and support network}
  \begin{itemize}
    \item Institute for Replication with Abel Brodeur coordinates the University of Ottawa hub.
    \item Support crew: Ghina Abdul Baki, Juan Pablo Aparicio, Bruno Barbarioli, Lenka Fiala, Derek Mikola, David Valenta.
    \item University of Ottawa hosts in person; virtual participants rely on Zoom. Email: instituteforreplication@gmail.com.
  \end{itemize}
\end{frame}

%-----------------------
\section{Study design and preparation}

\begin{frame}{Study design essentials}
  \begin{itemize}
    \item You will be randomly assigned to work either as a human-only participant or as part of the cyborg (AI-assisted) arm.
    \item Around 300 participants stratified by expertise tier and discipline tag.
    \item Randomized 1:1 within strata to AI-assisted access versus human-only control.
    \item Task pool spans Economics, Political Science, and Psychology with assignments balancing in- and out-of-discipline exposure.
  \end{itemize}
\end{frame}

\begin{frame}{Treatment arms and tiers}
  \begin{itemize}
    \item \textbf{Human control}. No external AI during the work window; document everything manually.
    \item \textbf{Cyborg arm}. ChatGPT Plus with Advanced Data Analysis, Deep Research, Agent Mode, and Codex CLI support; other AI tools are allowed if they document usage.
    \item Tiers from Undergraduate to Professor; we log discipline tags, coding experience, and AI familiarity for heterogeneity analyses.
  \end{itemize}
\end{frame}

\begin{frame}{Participant prep checklist}
  \begin{itemize}
    \item Complete this orientation and skim the reporting workbook (\href{https://github.com/I4Replication/AI-vertical/blob/main/Reports/Replication_Log_Referee_Template.xlsx}{GitHub template}).
    \item Bookmark the preregistered repository materials on \href{https://osf.io/dkfzt/}{OSF (osf.io/dkfzt)} for mirrored access.
    \item Accept the ChatGPT Team invite promptly.
    \item Confirm hardware and required software licenses (R/Stata/Python) before event day.
    \item Review the assignment email so you know your tier, discipline tag, arm, and team roster.
  \end{itemize}
\end{frame}

%-----------------------
\section{Event operations}

\begin{frame}{Event-day timeline and workflow}
  \begin{itemize}
    \item 8:45 local check-in or remote login; 9:00 shared Dropbox folder unlocks (OSF mirror provided for anyone without Dropbox) alongside the reporting sheet.
    \item Read instructions, note the focal result highlighted on the first page of the paper, and confirm you have every required file.
    \item Reproduce the assigned result, logging timestamps; audit code for major and minor errors.
    \item Run robustness checks and keep the reporting sheet—referee-report tab included—updated throughout the seven-hour window.
  \end{itemize}
\end{frame}

\begin{frame}{Deliverables, compliance, and support}
  \begin{itemize}
    \item Submit the reproduced result, error log, and reporting workbook by 16:00, plus qualitative notes if helpful.
    \item Control arm pledges no AI; AI arm completes a short end-of-day survey noting which AI tools they used and how often (no prompt logging required).
    \item Primary outcomes cover success, timing, error counts, and robustness; secondary outcomes review narratives and recommendations.
    \item Technical or design questions: email instituteforreplication@gmail.com.
  \end{itemize}
\end{frame}

\begin{frame}{What we mean by computational reproducibility}
  \begin{itemize}
    \item Anyone with the shared code, data, and instructions should be able to rerun the workflow and obtain the same focal result.
    \item Re-runs should execute end-to-end without manual tweaks, with scripts producing identical figures, tables, and statistics.
    \item Document external dependencies (software versions, seeds, APIs) so others can recreate the original computing environment.
  \end{itemize}
\end{frame}

\begin{frame}{Classifying coding issues}
  \begin{itemize}
    \item \textbf{Major errors}: Significantly change the numerical result, invalidate inference, or change conclusions.
    \item \textbf{Minor errors}: Issues that do not alter the reported outcome, inference or conclusions.
    \item Missing file paths, hard-coded directories, or absent packages are expected, do not treat them as coding errors.
  \end{itemize}
\end{frame}

\begin{frame}{Required robustness checks}
  \begin{itemize}
    \item Each team proposes and runs two targeted robustness checks tied to the assigned result.
    \item Prioritize checks that stress key assumptions (e.g., alternative specifications, sample trims, inference methods).
    \item Record the design, implementation status, and outcomes for each check in the reporting sheet.
  \end{itemize}
\end{frame}

\begin{frame}{Referee report deliverable}
  \begin{itemize}
    \item Use the referee-report tab inside the reporting sheet to summarize findings, robustness evidence, and recommendations.
    \item Focus on clarity: describe reproducibility outcomes, major/minor issues, and follow-up suggestions for the original authors.
    \item Keep evidence-linked: cite code cells, logs, or file names so the organizing team can audit quickly.
  \end{itemize}
\end{frame}

\begin{frame}{Post-event follow-up}
  \begin{itemize}
    \item Focus groups by treatment capture qualitative experience across arms.
    \item De-identified outputs enter the replication archive once the preregistration lock lifts.
    \item Participants receive summary results before journal submission and can provide feedback.
  \end{itemize}
\end{frame}

%-----------------------
\section{AI toolkit}

\begin{frame}{ChatGPT Plus toolkit}
  \begin{itemize}
    \item \feature{Advanced Data Analysis}. Run Python, upload files, and produce figures or tables in chat.
    \item \feature{Browsing \& Deep Research}. Reach current sources with citations and credibility checks.
    \item \feature{Custom GPTs \& Agent Mode}. Tailor assistants and supervise multistep execution inside your workflow.
  \end{itemize}
\end{frame}

%-----------------------
\begin{frame}{Q and A}
  \centering Thank you.
\end{frame}

\end{document}
