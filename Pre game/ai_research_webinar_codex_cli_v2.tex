
\documentclass[aspectratio=169,professionalfonts]{beamer}
\usetheme{default}

% Basic packages
\usepackage[utf8]{inputenc}
\usepackage{lmodern}
\usepackage{hyperref}
\usepackage{graphicx}
\usepackage{booktabs}
\usepackage{amsmath, amssymb}
\usepackage{array}
\usepackage{listings}
\usepackage{csquotes}
\usepackage{xspace}

% Beamer metadata
\title{Getting Research Done with ChatGPT Plus and Modern AI}
\subtitle{A one hour training for quantitative social sciences}
\author{Juan Pablo Posada Aparicio}
\institute{Institute for Replication, University of Ottawa}
\date{October 2025}

% Utilities
\newcommand{\feature}[1]{\textbf{#1}}
\newcommand{\demo}{\textbf{Demo. }}
\newcommand{\notepara}[1]{\small\textit{#1}}

\begin{document}

%-----------------------
\begin{frame}
  \titlepage
\end{frame}

%-----------------------
\begin{frame}{Session goals}
  \begin{itemize}
    \item Align on the Replication Games storyline, schedule, and roles before event day.
    \item Clarify randomization, treatment arms, and deliverables drawn from the preregistered design.
    \item Preview the AI training components that support literature, data, coding, writing, and review work.
    \item Reinforce responsible and reproducible practices across human-only and AI-assisted teams.
  \end{itemize}
\end{frame}

%-----------------------
\begin{frame}{Roadmap, about 60 minutes}
  \begin{itemize}
    \item 5 minutes. Pre-games context and storyline.
    \item 10 minutes. Study design, randomization, and assignments.
    \item 15 minutes. Event-day operations, deliverables, and support.
    \item 20 minutes. ChatGPT Plus feature tour across the research workflow.
    \item 5 minutes. Guardrails, reproducibility, and Q and A.
    \item 5 minutes. Codex overview at the end.
  \end{itemize}
\end{frame}

%-----------------------
\section{Pre-Games Orientation}

\begin{frame}{Pre-games storyline}
  \begin{itemize}
    \item Kick off the Replication Games cohort and align on the narrative before event day.
    \item Friendly human versus AI-augmented challenge tests speed, accuracy, and issue-spotting.
    \item We study vertical gaps across expertise tiers and horizontal gaps across disciplines.
  \end{itemize}
\end{frame}

\begin{frame}{Team and support network}
  \begin{itemize}
    \item Institute for Replication with Abel Brodeur coordinates the University of Ottawa hub.
    \item Support crew: Ghina Abdul Baki, Juan Pablo Aparicio, Bruno Barbarioli, Lenka Fiala, Derek Mikola, David Valenta.
    \item University of Ottawa hosts in person; virtual teams rely on Zoom and Slack. Email: instituteforreplication@gmail.com.
  \end{itemize}
\end{frame}

%-----------------------
\section{Study design and preparation}

\begin{frame}{Study design essentials}
  \begin{itemize}
    \item Around 300 participants stratified by expertise tier and discipline tag.
    \item Randomized 1:1 within strata to AI-assisted access versus human-only control.
    \item Task pool spans Economics, Political Science, and Psychology with assignments balancing in- and out-of-discipline exposure.
  \end{itemize}
\end{frame}

\begin{frame}{Treatment arms and tiers}
  \begin{itemize}
    \item \textbf{Human control}. No external AI during the work window; document everything manually.
    \item \textbf{Cyborg arm}. ChatGPT Plus with Advanced Data Analysis, Deep Research, Agent Mode, and Codex CLI support.
    \item Tiers from Undergraduate to Professor; we log discipline tags, coding experience, and AI familiarity for heterogeneity analyses.
  \end{itemize}
\end{frame}

\begin{frame}{Participant prep checklist}
  \begin{itemize}
    \item Complete this orientation and skim the reporting workbook (\href{https://github.com/I4Replication/AI-vertical/blob/main/Reports/Replication_Log_Referee_Template.xlsx}{GitHub template}).
    \item Accept the ChatGPT Team invite promptly; access stays scoped to event participants.
    \item Confirm hardware, software licenses (R/Stata/Python), VPN access, and data permissions before event day.
    \item Review the assignment email so you know your tier, discipline tag, arm, and team roster.
  \end{itemize}
\end{frame}

%-----------------------
\section{Event operations}

\begin{frame}{Event-day timeline and workflow}
  \begin{itemize}
    \item 8:45 local check-in or remote login; 9:00 Dropbox folder and reporting sheet unlock.
    \item Read instructions, identify the focal result, and catalog received files.
    \item Reproduce the assigned result, logging timestamps; audit code for major and minor errors.
    \item Run robustness checks and keep the reporting sheet updated throughout the seven-hour window.
  \end{itemize}
\end{frame}

\begin{frame}{Deliverables, compliance, and support}
  \begin{itemize}
    \item Submit the reproduced result, error log, and reporting workbook by 16:00, plus qualitative notes if helpful.
    \item Control arm pledges no AI; AI arm logs prompts, files, and outputs for internal usage analyses.
    \item Primary outcomes cover success, timing, error counts, and robustness; secondary outcomes review narratives and recommendations.
    \item Technical issues: contact Derek Mikola, Ghina Abdul Baki, Bruno Barbarioli, or Juan Pablo Aparicio. Design decisions: Abel Brodeur or Juan Pablo Aparicio.
  \end{itemize}
\end{frame}

\begin{frame}{Post-event follow-up}
  \begin{itemize}
    \item Focus groups by treatment capture qualitative experience across arms.
    \item De-identified outputs enter the replication archive once the preregistration lock lifts.
    \item Participants receive summary results before journal submission and can provide feedback.
  \end{itemize}
\end{frame}

%-----------------------
\section{AI toolkit}

\begin{frame}{LLM basics}
  \begin{itemize}
    \item Large language models predict the next token in context to follow instructions for language and code tasks.
    \item Strengths: structured generation, pattern completion, code drafting, translation. Weaknesses: hallucinations, stale knowledge, need for clear constraints.
    \item Your edge: provide roles, context, examples, and verification loops to steer outputs.
  \end{itemize}
\end{frame}

\begin{frame}{ChatGPT Plus toolkit}
  \begin{itemize}
    \item \feature{Advanced Data Analysis}. Run Python, upload files, and produce figures or tables in chat.
    \item \feature{Browsing \& Deep Research}. Reach current sources with citations and credibility checks.
    \item \feature{Custom GPTs \& Agent Mode}. Tailor assistants and supervise multistep execution inside your workflow.
  \end{itemize}
\end{frame}

\begin{frame}{Research workflow highlights}
  \begin{itemize}
    \item Literature scaffolds surface theories, debates, data sources, and structured reference tables.
    \item Data documentation drafts dictionaries, README structures, and polished tables from messy notes.
    \item Analysis support covers exploratory summaries, cleaning scripts, reusable helpers, and visualization guidance.
    \item Writing and review assistance interprets results, drafts abstracts, flags causal language risks, and generates referee-style feedback.
  \end{itemize}
\end{frame}

\begin{frame}{Advanced features in practice}
  \begin{itemize}
    \item Deep Research orchestrates medium-depth web reviews with explicit credibility and coverage criteria.
    \item Agent Mode sequences browsing, analysis, and drafting steps while keeping humans in the approval loop.
    \item Works best on scoped, auditable tasks where intermediate outputs remain inspectable.
  \end{itemize}
\end{frame}

\begin{frame}{Guardrails and checkpoints}
  \begin{itemize}
    \item Track prompts, data inputs, timestamps, and versions so provenance stays transparent.
    \item Verify claims against trusted sources and store scripts or logs with version history.
    \item Keep workflows reproducible and aligned with data governance, disclosure, and consent policies.
  \end{itemize}
\end{frame}

%-----------------------
\section{Codex overview}

\begin{frame}{Codex overview}
  \begin{itemize}
    \item Codex is OpenAI's coding agent that works inside git-managed projects alongside your local tools.
    \item Natural-language instructions drive code exploration, edits, and command execution with human approvals.
    \item Terminal-first workflow surfaces plans, diffs, and checkpoints so changes stay aligned with existing practices.
  \end{itemize}
\end{frame}

%-----------------------
\begin{frame}{Q and A}
  \centering Thank you.
\end{frame}

\end{document}
